\documentclass{IEEEtran}

\begin{document}

\title{A Survey of XAI Methods\\and Alignment with Autonmous Vehicles}
\author{
    \IEEEauthorblockN{David Kaczynski}
    \\\IEEEauthorblockA{University of Michigan}
}
\maketitle

\begin{abstract}
This is my abstract.  So, I've been looking a lot at the interpretation and explanation of black box machine learning methods, namely, neural networks, because they're mysterious, fascinating, and a little bit scary.  There are lots of options available for tools for developing neural networks, but the workflow of a data scientist is pretty well defined:  data preparation, feature extraction, dimensionality reduction, training, testing, and analysis of results.  We are specifically going to be focusing on a specific set of methods for analyzing results known as XAI, eXplainable Artificial intelligence.  In this paper, we explore the history and background that has popularized the use of black box machine learning methods, their relationships with society, the problems that XAI hopes to solve, and the challenges it faces in doing so.  Then, we take our exploration of XAI a step further by aligning its use cases with the development and use of autonomous vehicle technology.
\end{abstract}

\section{Introduction}

Machine learning and artificial intelligence (AI/ML) have gained mainstream attention as consumer products featuring these technologies permeate into everyday life.  Healthcare, transportation, supply chains, stock markets, social media, national security, genetic engineering, political science, and smart homes are all applying AI/ML to make faster, more accurate decisions and to automate tasks that previously required a human expert [citations needed].  As a wider audience is exposed to AI/ML, new questions are on the tips of people's tongues, like "what is the difference between artificial intelligence and machine learning?" and "do I need to be worried about computers making decisions that directly impact my health, security, and privacy?".

While artificial intelligence is a more familiar term and has a broader, amorphous definition, machine learning is a branch of AI that can be definid relatively explicitely.  Machine learning is the algorithmic processing of training data to create a computer program that can be used for repeatable tasks, such as making decisions or extracting insights from data.  In this sense, the computer can said to be "learning" by looking at existing data to create a model that can be applied to new data.  Some methods, such linear regression, are old news scooby doo.  Even the hot topic of neural networks was introduced while you were in grade school.  What *is* new is the availability of data, the raging locomotive research interest, and powerful GPUs and distributed commodity hardware to actually perform large scale AI/ML activity.  Despite AI being directly in the name, the field of XAI typically is more focused on so-called "black box" machine learning models, such as deep neural networks.  


Deep learning is a relatively new set of methods of machine learning based off an older concept called neural networks.  Thanks to advancement in GPU technology, cheaper and faster processing of linear equations has become affordable and available to a wider audience of consumers, including researchers, hobbyists, and professionals.  They excel and these things but not at those things, but hey, they're even getting better at those things like creeping us out with people faces.

Love it or hate it, AI/ML are like the Patriots:  they're here, they're winning, and there's nothing you can do about it.  Our lives are on the line.  Seriously.  Medical diagnosis, autonomous vehicles, personal and private details from the IoT...for safety, trust, and progress, XAI is srs, srsly.  


\subsection{Organization of Paper}

This paper is organized into the following sections.

\begin{itemize}
    \item\textit{Background}: Specifically, what are existing methods of XAI?  Also, there are some other pretty darn thorough surveys on XAI.

    \item\textit{Use Cases}: Methods of XAI can be used to the benefit users with a wide variety of backgrounds.  We identify four general use cases for data scientists and consumers.

    \item\textit{Alignment with Autonomous Vehicles}: The perspectives of both engineers and consumers of autonomous vehicles (AV) are considered as the relationship between XAI and AV are explored.

    \item\textit{Challenges}: XAI is a relatively young field, lacking formality, and there are several legitimate criticisms against it.  It is important to recognize the shortcomings of XAI and assess the current feasibility and progress in overcoming its obstacles.

    \item\textit{Conclusion and Future Research}: A summary of the alignment between XAI and AV is presented along with an outline of how identified obstacles may be approached.
\end{itemize}

\section{Background}

XAI is the interpretation and explanation of machine learning models.  In order to go further, the concepts of "interpretation" and "explanation" warrant a more formal definition:

\begin{itemize}
    \item an \textbf{interpretation} is the mapping of an abstract concept (e.g. a predicted class) into a domain that the human can make sense of, and
    \item an \textbf{explanation} is the collection of features of the interpretable domain, that have contributed for a given example to produce a decision (e.g. classification or regression) \cite{MONTAVON20181}.
\end{itemize}

Interpretations are often presented in the form of a heat map layer on an image, natural language generated to either describe a decision or to describe a boundary of what a model can or cannot do, or an easier-to-explain model, such as a decision tree.  An explanation is the relationship between some aspect of machine learning to something that is interpretable.  The ambiguity and lack of formality around the concept of an explanation is discussed further in \ref{sec:Challenges} Challenges in XAI.

\subsection{Visualization Techniques}

Introductory paper of LRP \cite{10.1371/journal.pone.0130140}

Activation Maximization \cite{Nguyen2016}

\subsection{Verbalization of Explanations}
Verbalization of CNN decisions \cite{10.1007/978-3-319-46493-0_1}

An introduction of counter factual explanations as a method of explaining decisions without interpreting "black box" internals \cite{DBLP:journals/corr/abs-1711-00399}

\subsection{Data Provenance}

Data provenance to trace decisions and analysis to the raw input data

Calculating the contribution of individual training samples to a trained model's decision \cite{Ma2017}

\subsection{Model Induction}

I'll need to put something here

\subsection{Existing Surveys in XAI}

\begin{itemize}
    \item Comprehensive survey of visualizations as XAI techniques \cite{Hohman2018}
    \item Methods of LRP \& heatmaps and their various use cases in research\cite{MONTAVON20181}
    \item A comprehensive survey of a variety of XAI techniques, includes sensitivity analysis (like heatmapping), rule and model extraction, activation maximization, and more \cite{Guidotti:2018:SME:3271482.3236009}, but it doesn't mention verbalization as a technique
    \item Network graphs plotting relationships between topics and sub-topics related to XAI \cite{Abdul:2018:TTE:3173574.3174156}
    \item Another broad survey of techniques in the field of XAI along with a discussion on its history and fundamental concepts, but again, no mention of verbalization as a technique\cite{Adadi2018}
\end{itemize}

\section{Use Cases} \label{sec:UseCases}

We've layed out various methods of XAI, but the question still remains of how do these methods provide value to anyone.  Here, we explore the various users who interface with XAI and what value is provided.

\subsection{As a data scientist, I can use the explanation of model decisions to identify opportunities to improve the training dataset and make the model more robust}

\begin{itemize}
    \item Montavan et al provide multiple real examples of researchers using a popular XAI method called Layerwise Relevance Propogation (LRP) for convolutional neural networks \cite{MONTAVON20181}.  In one example, it was discovered that a computer vision classifier was classifying horses based on a watermark that was present in the training dataset.
    \item The image of a dog was misclassified as a wolf \cite{Ribeiro:2016:WIT:2939672.2939778}.  Feature relevancy revealed that the snowy ground around the dog was deemed as the most relevant input pixels.  The model can be improved by adding images of dogs with snowy backdrops to the training dataset.
    \item Liu et al focus on interpreting the decisions of a model used to identify cybersecurity threats \cite{Liu:2018:ADM:3219819.3220027}.  The researchers interpret the model's decisions that failed to successfully label threats.  The interpretations help the researchers identify how to generate specific perterbations in the training data that improve the training of the model against previously misidentified threats.  The accuracy of the model improves.
    \item LAMP is a tool that can trace the decision of a trained model to the importance that individual samples had on that decision. \cite{Ma2017}
\end{itemize}

\subsection{As a potential consumer of an AI/ML product, such as an autonomous vehicle or virtual assistant, an explanation of how the product is making its decisions will improve my trust in the product}

\begin{itemize}
    \item Human subjects are asked about how much they trust a classifier before and after its explanations are made available \cite{Ribeiro:2016:WIT:2939672.2939778}.  
    \item Decisions from autonomous vehicles and advanced driver assistance systems (ADAS) may be explained using XAI methods to answer the questions "How?" and "Why?".  Koo et al measure both the emotional impact and impact in making safer decisions when drivers are provided answers to these questions \cite{Koo2015}
\end{itemize}

\subsection{As a provider of an AI/ML product, I am responsible for providing an explanation for how my product makes its decisions in the case of being the subject of an investigation or defendant in a lawsuit}

\begin{itemize}
    \item Human subjects conduct fictitious banking scenarios, both with and without descrimination-aware data mining (DADM) tools \cite{Berendt2014}.  The accuracy and presence of discrimination in decisions was compared across tools.
    \item Autonomous vehicles and vehicles with ADAS from Tesla, Google, and GM Cruise have been involved with numerous traffic incidencts, ranging from trivial to fatal \cite{Read2016} \cite{Tesla2018} \cite{Ackerman2016} \cite{Bhavsar2017} 
    \item Researchers discuss non-technical challenges and high-level technical challenges in investigating transparent model design in private industry \cite{Veale:2018:FAD:3173574.3174014}
    \item In the criminal justice system, judges and parole boards may apply predictive models as tools in making decisions.  Racial discrimination has been uncovered in such tools \cite{Wexler.2017} \cite{Angwin2016}.
    \item A magazine article describes the GPDR's "right to explanation" and includes a survey of research in identifying and rectifying descrimination \cite{Goodman2017EuropeanUR}
\end{itemize}

\subsection{As a member of a team of data scientists, I want to improve collaboration and reduce the duplication of effort by being able to effectively store, query, and trace how and on what data our  AI/ML models were trained, and, potentially, be able to trace a decision to the relevancy of each training item}

\begin{itemize}
    \item Researchers develop a tool on the distributed computation framework Spark that traces individual records through transformations and their relationship to aggregated or dervied values  \cite{Interlandi2017}.
    \item A team at from Amazon developed a platform and storage schema for effectively persisting and querying the activity of heterogenous ML tools \cite{Schelter2017}, providing the ability to replicate training of models and a platform for conducting analytics across development of ML models.
\end{itemize}

\section{Alignment with Autonomous Vehicles}\label{sec:Alignment}

The use cases defined in \ref{sec:UseCases} Use Cases span users with a wide variety of backgrounds.

\section{Challenges in XAI} \label{sec:Challenges}

\begin{itemize}
    \item Feature relevancy can be compromised by using a mask on the input that is virtually undetectable to the human eye \cite{DBLP:journals/corr/abs-1812-00891}.

    \item The concept of an explanation is subjective; there is no quantitative way of saying "yes, this decision is explained" or "no, this decision is not explained." \cite{Bibal2016}

    \item Due to the diverse audience of XAI, an explanation may be too technical or not technical enough to people of different backgrounds.
    
    \item Private industry is rarely incentivized to expose algorithms and ML models to researchers \cite{Veale:2018:FAD:3173574.3174014}.
\end{itemize}

\section{Conclusion and Future Research}

Apply XAI to \ref{sec:Alignment} Alignment with Autonomous Vehicles and also identify \ref{sec:Challenges} Challenges in XAI that may be worked towards.

\bibliography{references}
\bibliographystyle{ieeetr} 

\end{document}