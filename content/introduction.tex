\section{Introduction}

\subsection{History and Concepts}

Machine learning and artificial intelligence (AI/ML) have gained mainstream attention as consumer products featuring these technologies permeate into everyday life.  Healthcare, transportation, supply chains, stock markets, social media, national security, genetic engineering, political science, and smart homes are all applying AI/ML to make faster, more accurate decisions and to automate tasks that previously required a human expert [add citations here].  As a wider audience is exposed to AI/ML, new questions are on the tips of people's tongues, like "what is the difference between artificial intelligence and machine learning?" and "do I need to be worried about computers making decisions that directly impact my health, security, and privacy?".

Artificial intelligence is a broad term with an amorphous definition that changes over time, but machine learning can be more explicitly defined: machine learning is the algorithmic processing of training data to create a computer program that can be used for repeatable tasks, such as making decisions or extracting insights from data.  In this sense, the computer can said to be "learning" by looking at existing data and by creating a model that can be applied to new data.  While the term "machine learning" has been added to the English lexicon only recently, the foundation of machine learning was paved as far back as the 19th century.  Linear regression has its roots in the work of mathematicians Legendre and Gausse in the early 1800s [add citation], and even the hot topic of training neural networks was published in the 1970s \cite{Werbos1974}.  The catalyst for bringing these methods to the forefront of modern methods of AI and automation is the combination of the broad availability of data and powerful data centers of compute resources to actually perform large scale AI/ML activity.  Despite "AI" being directly in the name of XAI, the field is typically more focused on so-called "black box" machine learning models, such as deep neural networks.

Deep learning is a relatively new branch of machine learning methods based off the older concept of neural networks.  Thanks to advancement in GPU technology, the training of deeper, more complex neural networks has become affordable and available to common individuals, such as researchers, hobbyists, and data science professionals.  These deep neural networks (DNN) excel at optimizing the relationship between input and output variables from the training data without any input from a human expert describing rules or conditions.  This process of training a DNN inevitably creates a model whose internals are not able to be  understood via human inspection.  While the internals of the model are opaque, the optimized relationship that the model learned from the training data is able to quickly make decisions without ever having received any direct rules or instructions, making it ideal in situations like computer vision and natural language processing where explicit rules are difficult to define.

Through progress and controversy, artificial intelligence and machine learning are interfacing with more people in more ways every year, and there is a general awareness being raised about the legitimate concerns of the trustworthiness and accountability of these technologies.  Medical diagnosis, autonomous vehicles, and national security are just a few examples of ways in which machine learning can directly impact the health and well being of individuals' lives.  XAI is a new tool in the frontier of establishing traceable, trustworthy, and accountable decision systems.

*** This section should likely have something about autonomous vehicles in society ***

\subsection{Organization of Paper}

This paper is organized into the following sections.

\begin{itemize}
    \item\textit{Background}: Specifically, what are existing methods of XAI?  Also, there are existing literature surveys with their own perspectives.

    \item\textit{Use Cases}: Methods of XAI can be used to benefit users with a wide variety of backgrounds.  We identify four general use cases for data scientists, consumers, and auditors of AI/ML systems.

    \item\textit{Alignment with Autonomous Vehicles}: The perspectives of both engineers and consumers of autonomous vehicles (AV) are considered as the relationship between XAI and AV are explored.

    \item\textit{Challenges}: XAI is a relatively young field, lacking formality, and there are several legitimate criticisms against it.  It is important to recognize the shortcomings of XAI and assess the current feasibility and progress in overcoming its obstacles.

    \item\textit{Conclusion and Future Research}: A summary of the alignment between XAI and AV is presented along with an outline of how identified obstacles may be approached in the future.
\end{itemize}
